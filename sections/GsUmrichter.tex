\section{Gleichstromumrichter}
Ein Gleichstromumrichter dient zur Änderung von: \textbf{Polarität, Spannung, Strom}.\newline
\vspace{-0.2cm}
\[ \tau=\frac{L_1}{R_1} \qquad T_s=T_{on} + T_{off}\]
\vspace{-1cm}
%TODO V1 in V_in umbenennen
\subsection{Buck-Converter}
\begin{minipage}{0.75\linewidth}
    Tiefsetzsteller (Buck-Converter) $U_a < U_e  $\newline
    \textbf{Übung 6 Gleichstrom Umrichter $ \rightarrow $ Buck-Converter}\newline
    DutyCycle $ D = \frac{V_{out}}{V_1} \qquad T_{on} = DT_s \qquad T_{off} = (1 -D)T_s $\newline
    \renewcommand{\arraystretch}{2}
    \begin{tabular}{p{9cm} p{3cm}}
        $ V_1 = i_L\cdot R_1+L_1\cdot\frac{\diff i_L}{\diff t} $ &
        $ 0<t<T_{on} $
        \\  
        $ 0 = i_L\cdot R_1+L_1\cdot\frac{\diff i_L}{\diff t}$ & $T_{on}<t<T_{s} $
        \\  
        $ i_L=\frac{V_1}{R_1}+ \frac{V_1}{R_1}\cdot \frac{e^{-\frac{T_{off}}{\tau}}-1}{1-e^{-\frac{T_{s}}{\tau}}}\cdot e^{-\frac{t}{\tau}}$ &
        $ 0<t<T_{on}  $
        \\  
        $ i_L=\frac{V_1}{R_1}\cdot \frac{1-e^{-\frac{T_{on}}{\tau}}}{1-e^{-\frac{T_{s}}{\tau}}}\cdot e^{-\frac{t-T_{on}}{\tau}}$ &
        $ T_{on}\leq t \leq T_{s}  $
        \\ 
        $ i_{Lmax} = i_L(T_{on}) \qquad i_{Lmin} = i_L(0) = i_L(T_s) $    
        & \\ 
        $ T_{off}=-\tau \cdot ln\frac{i_{Lmin}}{i_{Lmax}}= -\frac{L_1}{R_1}\cdot ln\frac{i_{Lmin}}{i_{Lmax}} $
        & \\    
        $ T_{on}=-\tau \cdot ln\left(\dfrac{\frac{1}{i_{Lmax}}\cdot\frac{V_1}{R_1}-1}{\frac{1}{i_{Lmax}}\cdot \frac{V_1}{R_1}-e^{-\frac{T_off}{\tau}}}\right) $
        & \\ 
        $ V_{out}=V_{in}\cdot \frac{T_{on}}{T_{on}+T_{off}}-V_D\cdot\frac{T_{off}}{T_{on}+T_{off}} \approx V_{in}\cdot\frac{T_{on}}{T_s} $
        & \\ 
    \end{tabular}
    \newline
    
\end{minipage}
\begin{minipage}{0.25\linewidth}
    \includegraphics[width=\linewidth]{images/BuckOnOff}
    \includegraphics[width=\linewidth]{images/BuckSwitch}
\end{minipage}
\subsubsection{Lückbetrieb}
\begin{minipage}{0.75\linewidth}
    Wenn der Rippel der Amplitude zu hoch ist, wird die Spule komplett entladen bevor die Periode zu ende ist.

    \begin{tabular}{p{8cm} p{4cm}}
    $ T_{off} = T_{\delta}+T_{off2} $ &
    $ T_{\delta}$ = Diode leitet \newline
    $ T_{off2}$ = Diode sperrt
    \\  
    
    $a' = a (U\frac{V_1}{V_{out}}-1)$&
    $ a = \frac{T_{on}}{T_S} \qquad  a' = \frac{T_{\delta}}{T_S}$
    \\
    
    $ \varDelta i_2 = \frac{V_1-V_{out}}{L}\cdot a \cdot T_S $&
    $ 0 < t < T_{on} $ steigend
    \\
    
    $ \varDelta i_2 = \frac{V_{out}}{L}\cdot a' \cdot T_S $&
    $ T_{on} < t < T_{\delta} $ fallend
    \\
    
    $ \varDelta i_{2 \; max} = \frac{T_S\cdot V_{1}}{4L} $&
    für a = $\frac{1}{2} $
    \\
    
    $ \bar{i}_{2} = \frac{\varDelta i_2}{2}a\frac{V_1}{V_{out}} = a^2 \frac{T_s V_1}{2 L}\frac{V_1 - V_{out}}{V_{out}} $&
    \\
    
    $ V_{out}= V_1 \frac{1}{\frac{2L\cdot I_0}{D^2\cdot V_1 \cdot T_s}+1} $&
    \\
         
    \end{tabular}
\end{minipage}
\renewcommand{\arraystretch}{1}
\begin{minipage}{0.25\linewidth}
    \includegraphics[width=\linewidth]{images/BuckSwitchLueck}
\end{minipage}

\begin{minipage}{0.4\linewidth}
    \includegraphics[width=\linewidth]{images/BuckLueckGrenze}
\end{minipage}
\begin{minipage}{0.6\linewidth}
    \includegraphics[width=\linewidth]{images/BuckLueckBetrieb}
\end{minipage}
\clearpage

\subsection{Boost-Converter}
\begin{minipage}{0.75\linewidth}
    Hochsetzsteller (Boost-Converter) $U_a > U_e  $\newline
    $ D=1-\frac{V_1}{V_{out}} $\newline
    \renewcommand{\arraystretch}{2}
    \begin{tabular}{p{9cm} p{3cm}}
        $ V_{Out}=V_{in}\cdot \left(1+\frac{T_{on}}{T_{off}} \right)$&
        \\
        $ \varDelta i_L = \frac{U_0}{L}\cdot T_E=-\frac{U_0  U_d}{L}\cdot(T-T_E)$&\\
        
        $ u_L = U_0 - R_L i_L$&
        $ 0<t<T_{on} $\\
        
        $ u_L = U_0 - R_L i_L -V_{out}$&
        $T_{on}<t<T_{s} $\\
        
        $ \bar{u}_L=\frac{1}{T_s}\int\limits_{0}^{T_{on}}(U_0 - R_L i_L)\diff t + \frac{1}{T_s}\int\limits_{T_{on}}^{T_s}(U_0 - R_L i_L)\diff t $&
        \\
        
        $ \eta(a) = \frac{1}{1+\frac{R_L}{R_1}\frac{1}{(1-a)^2}} $ &
        $ a=\frac{T_{on}}{T_s} $\\
                     
    \end{tabular}
    \renewcommand{\arraystretch}{1}
\end{minipage}
\begin{minipage}{0.25\linewidth}
    \includegraphics[width=\linewidth]{images/BoostOnOff}
    \includegraphics[width=\linewidth]{images/BoostSwitch}
\end{minipage}
\subsubsection{Lückbetrieb}
\begin{minipage}{0.75\linewidth}
        todo\newline
    
    \begin{tabular}{p{9cm} p{3cm}}
        $ I_{L\;Max} = \frac{V_{in}\cdot T_s \cdot D}{L} $ &
        $ 0<t<T_{on} $
        \\  
        
        $ I_{L\;Max} + \frac{V_{in}-V_{out} T_{\delta}}{L} = 0 $ &
        $ T_{on}<t<T_{\delta} $
        \\  
        
        $ I_{out}=\frac{V_{in}^2\cdot D^2\cdot T_s}{2L(V_{out}-V_{in})} $&
        \\
        
    \end{tabular}
\end{minipage}
\begin{minipage}{0.25\linewidth}
    \includegraphics[width=\linewidth]{images/BoostSwitchLueck}
\end{minipage}

\subsection{Inverse-Converter}
\begin{minipage}{0.75\linewidth}
Inverswandler, Umkehrung der Polarität\newline
\[ V_{out}=-L\cdot \frac{\varDelta I_L }{\varDelta t} \overbrace{=}^{eingeschwungen}V_L \cdot \frac{T_{on}}{T_{off}} \]
\end{minipage}
\begin{minipage}{0.25\linewidth}
    \vspace{-1cm}
    \includegraphics[width=\linewidth]{images/InverseOnOff}
    \includegraphics[width=\linewidth]{images/InverseSwitch}
\end{minipage}
\subsubsection{Lückbetrieb}
\begin{minipage}{0.75\linewidth}
    $ D= \frac{V_{off}}{V_{off}+V_{on}} $\newline
    todo\newline
    
    \begin{tabular}{p{9cm} p{3cm}}
        $ I_{L\;Max} = \frac{V_{in}\cdot T_s \cdot D}{L} $ &
        $ 0<t<T_{on} $
        \\  

    \end{tabular}
\includegraphics[width=0.2\linewidth]{images/InverseLueckGrenze}
\end{minipage}
\begin{minipage}{0.25\linewidth}
    \includegraphics[width=\linewidth]{images/InverseSwitchLueck}
\end{minipage}
\clearpage

\subsection{Gleichstromschalter/Gleichstromsteller}
\subsubsection{Gleichstromschalter}
\begin{minipage}{0.5\linewidth}
    \textbf{Nur Einschalten}\newline
    \[ U_1=(L + L_{\sigma})\cdot \frac{\diff i_L}{\diff t}+R\cdot i_L \]
    \[ i_L(t)=\frac{U_1}{R}\cdot(1-e^{-\frac{t-t_{on}}{\tau}})\]
\end{minipage}
\begin{minipage}{0.4\linewidth}
    \includegraphics[width=1.2\linewidth]{images/GsSchalterOn}
\end{minipage}

\begin{minipage}{0.5\linewidth}
\textbf{Ein- und Ausschalten}\newline
\[ U_1=(L + L_{\sigma})\cdot \frac{\diff i_L}{\diff t}+R\cdot i_L \qquad t_{on} \leq t \leq t_{off}\]
\[ 0=L\cdot \frac{\diff i_L}{\diff t}+ R\cdot i_L \qquad \qquad \quad t_{off}\leq t \]
\[ i_L(t)=\frac{U_1}{R}\cdot(1-e^{-\frac{t-t_{on}}{\tau}}) \qquad \quad t_{on} \leq t \leq t_{off}\]
\[ i_L(t)=\frac{U_1}{R}\cdot e^{-\frac{t-t_{off}}{\tau}} \qquad \quad \qquad t_{off}\leq t \]
\end{minipage}
\begin{minipage}{0.4\linewidth}
    \includegraphics[width=1.2\linewidth]{images/GsSchalterOnOff}
\end{minipage}

\subsubsection{Gleichstromsteller}
\begin{minipage}{0.5\linewidth}
\[ U_{2AV}=\frac{1}{t_{on}+t_{off}}\int_{0}^{t_e}U_1\cdot \diff t = \frac{t_{on}}{t_{on}+t_{off}}U_1 \]
\end{minipage}
\begin{minipage}{0.4\linewidth}
\includegraphics[width=1.2\linewidth]{images/GsSteller}
\end{minipage}

\begin{minipage}{0.5\linewidth}
    \includegraphics[width=0.8\linewidth]{images/GsStellerPuls}
\end{minipage}
\begin{minipage}{0.5\linewidth}
    \includegraphics[width=0.8\linewidth]{images/GsStellerFreq}
\end{minipage}

\textbf{Ein-Quadranten-Betrieb}\newline
\begin{minipage}{0.7\linewidth}
    \begin{longtable}{   p{.50\textwidth}  p{.6\textwidth} }
         $U_{m2} = \frac{T_E}{T}\cdot U_D = F_p\cdot T_E \cdot U_D$ &
         $U_{m2}$ = Mittelwert Ausgangsspg.\newline
         $T_e $ = Einschaltzeit \newline
         $T$ = Periodendauer \newline
         $U_D$ = Zwischenkreisspg. \newline
         $f_q$ = Freq
         \\  
         
         $\varDelta i_2 = 2 \alpha\frac{U_D}{L}T_E(1-\frac{T_E}{T})$ &
         $\alpha$ = 0.5 \quad 1Q-Betrieb \newline
         $\alpha$ = 1 \quad MehrQ-Betrieb
         \\
         
         $ P = U_D I_{m2}\frac{T_E}{T} $&
         $i_2$ =Ausgangsstrom\newline
         $i_{m2}$ = Mittelwert Ausgansstr.\newline
         $L$ = ind. der Last \newline
         $P$ = Ausgangsleistung
         \\
         
         $U_{ac\; 2}=\sqrt{U_2^2-U_{m2}^2}= U_D\sqrt{\frac{T_E}{T}(1-\frac{T_E}{T})} $ &
         $U_{ac2}$= Wechselstromkomponente von U ist\newline
         maximal bei Tastgrad 50\% \\        
    \end{longtable}
\end{minipage}
\begin{minipage}{0.3\linewidth}
    \includegraphics[width=0.8\linewidth]{images/GsSteller1Q}
\end{minipage}

\clearpage