\thispagestyle{empty}
\setcounter{page}{0} %Set PageNumber to 0
{\huge README }
\section*{Beschreibung}
Formelsammlung für Leistungselektronik 1 auf Grundlage der Vorlesung HS 16 von Prof. Dr.Jasmin Smajic \newline
\textbf{Die Formelsammlung ist noch nicht vollständig. Ergänzungen über GitHub sind erwünscht.}\newline
Bei Korrekturen oder Ergänzungen wendet euch an einen der Mitwirkenden.
\subsection*{OrCad}
ORCAD Dateien und Simulationen zu den Übungen findet ihr auf OneDrive: \newline
\url{https://1drv.ms/f/s!AigDIY3mrirZhqxYqS0gULPbaWLCfw}\newline
\textbf{Installation}\newline
\begin{itemize}
    \item gehe zu  \textbackslash\textbackslash hsr.ch\textbackslash root\textbackslash alg\textbackslash software\textbackslash Elektrotechnik\textbackslash LeistEl\textbackslash orcad16.6
    \item Rechtsklick auf \" OrcadSetup-starten-Run as administrator \" und \" Run as Administrator \" auswählen
    \item Orcad wird installiert
    \item Nach der Installation dem Hotfix-Link folgen und von dort aus den Hotfix installieren
\end{itemize}

\section*{Modulschlussprüfung}
Prüfungsstoff ist der gesamte LeistEL-Vorlesungsinhalt des HS2016 einschliesslich aller UE + P Lerninhalte.\newline
Als Hilfsmittel für die Modulschlussprüfung sind die Vorlesungen,\newline
UE-Aufgaben und eigenen Praktikumsaufzeichnungen sowie der Taschenrechner erlaubt.

\subsection*{Plan und Lerninhalte}
{\scriptsize 
    \begin{itemize}
        \item Elektronische Schalter (Leistungshalbleiter) und passive Stromrichterkomponenten 
        \item Aufbau und Arbeitsweise netzgeführter Stromrichter 
        \item Wechsel- und Drehstromsteller 
        \item Fremdgeführte Stromrichter
        \item Selbstgeführte Stromrichter
        \item Umrichter
        \item Steuerung und Regelung von Stromrichtern
        \item Einsatz von Stromrichtern in der Energietechnik
        \item Elektromagnetische Verträglichkeit und Netzrückwirkungen
    \end{itemize}
}
\vfill
\section*{Contributors}
\begin{tabular}{ll}
    Luca Mazzoleni& luca.mazzoleni@hsr.ch \\ 
\end{tabular} 

{\scriptsize 
    \section*{License}
    \textbf{Creative Commons BY-NC-SA 3.0}
    
    Sie dürfen:
    \begin{itemize}
        \item Das Werk bzw. den Inhalt vervielfältigen, verbreiten und öffentlich
        zugänglich machen.
        \item Abwandlungen und Bearbeitungen des Werkes bzw. Inhaltes anfertigen.
    \end{itemize}
    Zu den folgenden Bedingungen:
    \begin{itemize}
        \item Namensnennung: Sie müssen den Namen des Autors/Rechteinhabers in der von ihm
        festgelegten Weise nennen.
        \item Keine kommerzielle Nutzung: Dieses Werk bzw. dieser Inhalt darf nicht für
        kommerzielle Zwecke verwendet werden.
        \item  Weitergabe unter gleichen Bedingungen: Wenn Sie das lizenzierte Werk bzw. den
        lizenzierten Inhalt bearbeiten oder in anderer Weise erkennbar als Grundlage
        für eigenes Schaffen verwenden, dürfen Sie die daraufhin neu entstandenen
        Werke bzw. Inhalte nur unter Verwendung von Lizenzbedingungen weitergeben,
        die mit denen dieses Lizenzvertrages identisch oder vergleichbar sind.
    \end{itemize}
    Weitere Details: http://creativecommons.org/licenses/by-nc-sa/3.0/ch/
}
%If we meet some day, 
%and you think this stuff is worth it, you can buy me a beer in return.
\clearpage
\pagenumbering{arabic}% Arabic page numbers (and reset to 1)