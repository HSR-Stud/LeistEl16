\section{Ungesteuerter Gleichrichter}
\subsection{M1U}
\vspace{-0.5cm}
\begin{minipage}{0.4\linewidth}
    \includegraphics[width=\linewidth]{images/PrakUGM1}
\end{minipage}
\begin{minipage}{0.3\linewidth}
    \centering
   \includegraphics[width=0.7\linewidth]{images/PrakUGM1Kl1}
   \includegraphics[width=0.7\linewidth]{images/PrakUGM1Kl2}
\end{minipage}
\begin{minipage}{0.3\linewidth}
    \includegraphics[width=\linewidth]{images/UGM1OW} 
\end{minipage}
\newline

%\includegraphics[width=0.4\linewidth]{images/UGRM1U}
%\includegraphics[width=0.2\linewidth]{images/UGRM1US} \newline
\vspace{-0.3cm}
Die Diode wird als ideal betrachtet $ \rightarrow $ keine Schwellenspannung oder Innenwiderstand
	\renewcommand{\arraystretch}{1}
	\vspace{-0.2cm}
\begin{longtable}{| p{.3\textwidth} | p{.40\textwidth} | p{.25\textwidth} |} 
    \hline
    \textbf{Grundgleichungen}&
    \vspace{-0.5cm}
    \[ U_2 = U_D + U_R \]
    \[ U_R = I_2 \cdot R\]
    \[ U_{OUT} = \dfrac{\widehat{U}}{\pi}\]\vspace{-1cm}&
    \textbf{Durchlassrichtung}\newline
    $ 0 < \omega t < \pi $\newline
    $U_2 = U_R \qquad U_D = 0$\newline

    \textbf{Sperrichtung}\newline
    $ \pi < \omega t < 2\pi $\newline
    $ U_2=U_D \qquad U_R = 0 $\newline
    \\
    \hline
    
    \textbf{Wirkleistung der Last R}&
    \vspace{-0.5cm}
    \[ P=\frac{1}{2\pi} \int_{0}^{2\pi} p(\alpha) d\alpha = \dfrac{U_{R\;RMS}^2}{R} \] \vspace{-0.5cm}&
    \\ \hline
\end{longtable}
%
%Leistung = Momentanleistung des Sormes x momentanleistung der Spannung\\
%Leistung = Leistung bei trafoseite messen 1harm des stroms phasenverschiebung -> u i cos(phi) fourierreihen..
%Leistung = Irav* U1harm  
\vspace{-0.3cm}
\textbf{Oberwellen}\\[0.2cm]
\vspace{-1cm}
\begin{figure}[h]
	\begin{subfigure}[t]{0.5\linewidth}
		\includegraphics[width=\linewidth]{images/M1UR}
		\subcaption{R}
	\end{subfigure}
\begin{subfigure}[t]{0.5\linewidth}
	    \includegraphics[width=\linewidth]{images/M1URL}
	    \subcaption{R + L}
\end{subfigure}
\begin{subfigure}[t]{0.5\linewidth}
	    \includegraphics[width=\linewidth]{images/M1URLD}
	    \subcaption{R + L+ Freilaufdiode}
\end{subfigure}
\begin{subfigure}[t]{0.24\linewidth}
    \includegraphics[width=\linewidth]{images/M1URLKl}
    \subcaption{R + L Kennlinie}
\end{subfigure}
\begin{subfigure}[t]{0.25\linewidth}
    \includegraphics[width=\linewidth]{images/M1URCKl}
    \subcaption{R + C Kennlinie}
\end{subfigure}
\end{figure}
\vspace{-0.8cm}
\subsubsection{Rechnungsbeispiel}
\textbf{Übung 2 - Gleichrichter M1U}\newline
\renewcommand{\arraystretch}{1.3}
\begin{tabular}{ p{.3\textwidth}  p{.40\textwidth}  p{.25\textwidth}}
    Ausgangslage:&
    $ U_2= U_{2m}\cdot sin(2\pi ft)$&
    \\
    Mittelwert Spannung: &
    $ U_{R \; AV}= \frac{1}{2 \pi} \int\limits_{0}^{\pi} U_{2m}\cdot sin(\alpha)\diff\alpha=\frac{U_{2m}}{\pi} $ &
    $ \alpha=\omega t $
    \\
    
    Effektivwert Spannung:   &
    $ U_{R \; RMS}= \sqrt{\frac{1}{2 \pi}\int\limits_{0}^{\pi} U_{2m}^2\cdot sin(\alpha)^2\diff\alpha} = \frac{U_{2m}}{2} $ &
    \\ 
    
    Mittelwert Strom: &
    $ I_{R \; AV}=\frac{U_{R \; AV}}{R_L}= \frac{1}{\pi}\frac{U_{2m}}{R}= \frac{1}{\pi} I_{2m} $ &
    \\
    
    Effektivwert Strom: &
    $ I_{R \; RMS}=\frac{U_{R \; RMS}}{R}= \frac{U_{2m}}{2R}= \frac{I_{2m}}{2} $ &
    \\
    
    Wirkleistung R: &
    $ P= \frac{1}{2\pi}\int\limits_{0}^{\pi}\frac{u_R^2(\alpha)}{R} \diff \alpha = \frac{U_{R \; RMS}^2}{R}=\frac{1}{R}\frac{U_{2m}^2}{4} $&
    \\
    
    Wirkleistung Sekundärseite: &
    $P_2 = U_2 \cdot (I_2)_1 \cdot cos(\varphi)_1 = \frac{U_{2m}}{\sqrt{2}} \cdot \frac{b_1}{\sqrt{2} \cdot R}$&
    \\
\end{tabular}\vspace*{-1cm}
\renewcommand{\arraystretch}{1}

%===================================================================
\clearpage

\subsection{B2U}
\begin{minipage}{0.4\textwidth}
    \includegraphics[width=\linewidth]{images/PrakUGB2}
\end{minipage}
\begin{minipage}{0.25\linewidth}
    \centering
    \includegraphics[width=0.9\linewidth]{images/PrakUGB2Kl1}
    \includegraphics[width=0.9\linewidth]{images/PrakUGB2Kl2}
\end{minipage}
\begin{minipage}{0.35\linewidth}
    \includegraphics[width=\linewidth]{images/UGB2OW}
\end{minipage}\newline

Im Gegensatz zur M1U-Schaltung wird hier die negative Netzspannung zur Gleichrichtung genutzt.\newline
Die Schaltung wird oft mit Glättungskondensatoren betrieben.
\begin{longtable}{| p{.33\textwidth} | p{.40\textwidth} | p{.25\textwidth} |} 
    \hline
    \textbf{Grundgleichungen}&
    \[ \bar{U}_{OUT} = \dfrac{2\widehat{U}}{\pi}\]&\\
    \hline   
\end{longtable}
\subsubsection{Rechnungsbeispiel}
\textbf{Übung 3 - Gleichrichter B2U}\newline
\renewcommand{\arraystretch}{2}
\begin{tabular}{ p{.3\textwidth}  p{.40\textwidth}  p{.25\textwidth}}
    Ausgangslage:&
    $ U_2= U_{2m}\cdot sin(2\pi ft)$&
    \\
    Mittelwert Spannung: &
    $ U_{R \; AV}= \frac{1}{\pi} \int\limits_{0}^{\pi} U_{2m}\cdot sin(\alpha)\diff\alpha=\frac{2\cdot U_{2m}}{\pi} $ &
    $ \alpha=\omega t $
    \\
    
    Effektivwert Spannung:   &
    $ U_{R \; RMS}= \sqrt{\frac{1}{\pi}\int\limits_{0}^{\pi} U_{2m}^2\cdot sin(\alpha)^2\diff\alpha} = \frac{U_{2m}}{\sqrt{2}} $ &
    \\ 
    
    Mittelwert Strom: &
    $ I_{R \; AV}=\frac{U_{R \; AV}}{R_L}= \frac{2}{\pi}\frac{U_{2m}}{R}= \frac{2}{\pi} I_{2m} $ &
    \\
    
    Effektivwert Strom: &
    $ I_{R \; RMS}=\frac{U_{R \; RMS}}{R}= \frac{U_{2m}}{\sqrt{2}\cdot R}= \frac{I_{2m}}{\sqrt{2}} $ &
    \\
    
    Wirkleistung R: &
    $ P= \frac{1}{\pi}\int\limits_{0}^{\pi}\frac{u_R^2(\alpha)}{R}\diff \alpha = \frac{U_{R \; RMS}^2}{R}=\frac{1}{R}\frac{U_{2m}^2}{2} $&
    \\
    Wirkleistung Sekundärseite: &
    $P_2=U_2\cdot I_2=U_2\cdot \frac{U_2}{R}=U_2\cdot \frac{U_{2m}}{\sqrt{2}R}$&
    \\
\end{tabular}
\renewcommand{\arraystretch}{1}

%===================================================================
%\clearpage

\subsection{B6U}
\includegraphics[width=0.3\linewidth]{images/PrakUGB6}
\includegraphics[width=0.3\linewidth]{images/PrakUGB6Kl1}
\includegraphics[width=0.3\linewidth]{images/UGB6OW}\newline
\clearpage