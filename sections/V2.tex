\section{Diode}
Eine Diode besteht aus PN übergängen und ist deswegen ein nichtlineares Element:
\begin{multicols}{2}
     \begin{minipage}{\linewidth}
         \begin{tabular}{ll}
             $ U_{DBR} $    & Durchbruchspanung  \\ 
             $ U_D $        & Diffusionsspannung (0.7V Si)  \\ 
             $ U_F $        & Flussspannung \\ 
             $ U_R $        & Sperrspannung  \\ 
              $ i_F $       & Diffusionsstrom, Strom in Durchlassrichtung  \\ 
              $ i_R $       & Leckstrom, Strom in Sperrichtung \\ 
            \end{tabular} 
        \end{minipage}
        
         \includegraphics[width=0.8\linewidth]{images/kennlinieDiode}
\end{multicols}
\vspace{-1.5cm}
\begin{multicols}{2}
     \begin{minipage}{\linewidth}
         \begin{tabular}{ll}
             $ i_f ,\; u_F $& Durchlassrichtung\\
             $ U_{T0} $& Schwellenspannung\\
             $ r_f= \dfrac{\diff U_F}{\diff i_F} $& Differenzieller Durchlasswiderstand\\
            \end{tabular}       
    \end{minipage}
        
        \includegraphics[width=0.4\linewidth]{images/dDiodeKennlinie}
\end{multicols}
\vspace{-1.5cm}
\subsection{Ersatzschaltbild}
\begin{multicols}{4}
    \begin{minipage}{\linewidth}
        \textbf{Reale Diode} \raggedright \newline\newline
        \includegraphics[width=0.7\linewidth]{images/realeDiode}
    \end{minipage}
    
    \begin{minipage}{\linewidth}
        \textbf{Ideale Diode ($ D_1 $)} \raggedright \newline\newline
        \includegraphics[width=0.7\linewidth]{images/idealeDiode}
    \end{minipage}
    
    \begin{minipage}{\linewidth}
        \textbf{Diode $ D_1 $ mit der Schwellenspannung $  D_2 $} \raggedright
        \includegraphics[width=0.6\linewidth]{images/idealeDiodeSP}
    \end{minipage}
    
    \begin{minipage}{\linewidth}
        \textbf{Diode $ D_2 $ mit dem Durchlasswiderstand($ D_3 $)} \raggedright
        \includegraphics[width=0.6\linewidth]{images/idealeDiodeSPR}    
    \end{minipage}                       
\end{multicols}
\begin{multicols}{2}
\subsection{Grundformeln}
\begin{tabular}{ll}
    \textbf{Flussspannung}&\[ u_F = U_{T0} + i_F \cdot r_F \]\\
    \textbf{Momentanleistung}&\[ p(t)=u_F(t)\cdot i_F(t) \]\\
    \textbf{Verlustleistung}&\[ P_v=U_{TO}\cdot I_{FAV}+r_F\cdot I_{FRMS}^2 \]\\
    $ I_{FAV}$ & arithmetische Mittelwert von $ i_F $\\
    $ I_{FRMS} $ & Effektivwert von $ i_F $\\    
\end{tabular}

\hspace{0.5cm}\includegraphics[width=0.6\linewidth]{images/ESBDiode} 
\end{multicols}

\subsection{Schaltverhalten und Schaltverluste}
    \begin{minipage}{0.7\linewidth}
        \raggedright
        \textbf{Durchlassverzug:}\newline
        freie Ladungsträger müssen zuerst die Ladungsfrei Zone "füllen"\newline\newline
        \textbf{Sperrverzug}\newline
        freie Ladungsträge müssen zuerst das Gebiet des pn-Überganges freiräumen\newline\newline
        Diese Erscheinungen sind wichtig bei  $\nicefrac{\diff u}{\diff t} > 100 \nicefrac{V}{\mu s} \; und \; \nicefrac{\diff i}{\diff t} > 10 \nicefrac{A}{\mu s} $
        \begin{tabular}{ll}
            $ I_{RM} $&Maximalwert des Rückstroms\\
            $ U_{RM} $&Maximalwert der Rückspannung\\
        \end{tabular}
    \end{minipage}   
    \begin{minipage}{0.3\linewidth}
        \vspace{-0.8cm}
        \raggedleft
            \includegraphics[width=\linewidth]{images/idealeDiodeSS}           
            \includegraphics[width=\linewidth]{images/realeDiodeSS}
    \end{minipage}
\clearpage

\vspace{-0.5cm}
\section{Transistor}
\vspace{-0.2cm}
\subsection{Bipolarer Transistor}
\vspace{-0.2cm}
\subsubsection{Wirkungsprinzip}
Ein Bipolartransistor besteh aus drei dünnen dotierten Halbleiterschichten, d.h. aus zwei pn-Übergängen. Gemäss der Reihenfolge und dem Dotierungstyp der Schichtung werden Bipolartransistoren in npn- und pnp-Typen unterteilt.\newline Als Leistungstransistoren werden überwiegend npn- Transistoren in der Emmitter-Schaltung verwendet\newline

\begin{tabular}{ccc}
     \textbf{Wirkungsprinzip}&\textbf{Aufbau}&\textbf{Schaltzeichen}\\
     \includegraphics[width=0.35\linewidth]{images/npnTransistor}&
     \includegraphics[width=0.15\linewidth]{images/aufbautransnpn}&
     \includegraphics[width=0.25\linewidth]{images/esbtransnpn} \\
\end{tabular}

\subsubsection{Schaltverhalten}
\begin{minipage}{0.6\linewidth}
    \begin{wrapfigure}{r}{4cm}
        \includegraphics[width=\linewidth]{images/npnTransemitter}
    \end{wrapfigure}
    \raggedright
    \textbf{Im Sättigungsbreich} ist der Basisstrom so gross, dass sich in der Basiszone mehr Ladungsträger befinden als für den Kollektorstrom nötig ist.\newline\newline
    Die beiden pn-Übergänge sind in die Durchlassrichtung polarisiert.\newline
    $ U_{BE}>U_{CE} $ und $ U_{BC}>0 $\newline\newline
    Im Verstärkungsbereich gilt: $ \beta = \dfrac{I_C}{I_B} $\newline \newline
    \textbf{Im Schaltbetrieb} werden die Arbeitspunkte \textbf{I} (vorwärts sperrend) und \textbf{III} (Durchlassbetrieb -Sättigung) verwendet.
\end{minipage}
\begin{minipage}{0.4\linewidth}
    \includegraphics[width=0.8\linewidth]{images/npnTranskennlinie}
\end{minipage}
\subsubsection{Kennwerte}%TODO Leftaligned
\begin{tabularx}{0.5\linewidth}{l p{7cm}} 
    \textbf{$ U_{CES} $}&\textbf{Kollektor-Emitter-Sperrspannung}\\
    &Der höchstzulässige Wert der $ U_{CES} $bei Ansteuerung mit einer negativen $ U_{BE} $\\
    \textbf{$ U_{CE0} $}&\textbf{Kollektor-Emitter-Sperrspannung}\\%RICHTIG??
    &Der höchstzulässige Wert der $ U_{CE} $ bei offenem Basisanschluss\\
    \textbf{$ I_{CAVM} $}&\textbf{Kollektor-Dauergrenzstrom}\\
    &Der höchstzulässige Wert des Gleichstrom-Mittelwerts bei vorgegebener Temperatur\\
    \textbf{$ I_{CRM} $}&\textbf{periodischer Kollektor-Spitzenstrom}\\
    &der höchstzulässige Wert eines Pulsstromes mit angegebener Periodendauer und Einschaltdauer\\
\end{tabularx}
\begin{minipage}{0.5\linewidth}
    \includegraphics[width=\linewidth]{images/npnTransESV}
\end{minipage}

\begin{minipage}{0.5\linewidth}
    \subsubsection{Verluste}
    \begin{itemize}
        \item Einschaltverluste
        \item Ausschaltverluste
        \item Durchlassverluste
        \item Sperrverluste
    \end{itemize}
\end{minipage}
\begin{minipage}{0.5\linewidth}
    \includegraphics[width=\linewidth]{images/npnTransVerluste}
\end{minipage}
\clearpage

\vspace*{-1cm}
\subsection{Darlington-Transistoren}
\begin{wrapfigure}{r}{2cm}
    \includegraphics[width=\linewidth]{images/darlingtonSymbol}
\end{wrapfigure}
Der Stromverstärkungsfaktor der Leistungstransistoren ist relativ klein. Deswegen ist ein straker Basisstrom für diese Transistoren notwendig. Ein Darlington-Transistor löst diese Problem.

\begin{minipage}{0.6\linewidth}
    \subsubsection{Formeln}
    \vspace{-0.5cm}
    \[ \beta_1 = \dfrac{i_{C1}}{i_{B1}} \qquad \beta_2 = \dfrac{i_{C2}}{i_{B2}} \]    
    \[ i_{E1} = i_{C1}+i_{B1}=(1+\beta_1)i_{B1} = i_{B2} \]
    \[ i_{C2} = \beta_2 i_{B2} = \beta_2 i_{E1} = \beta_2 (1 + \beta_1)i_{B1}=\beta_{ges}i_{B1} \]
    \[ \beta_{ges} = \beta_2(1 + \beta_1) \approx \beta_1 \beta_2 \]    
\end{minipage}
\begin{minipage}{0.3\linewidth}
    \subsubsection{Aufbau}
    \includegraphics[width=\linewidth]{images/darlingtonaufbau}
\end{minipage}

\subsubsection{Vor und Nachteile}
\vspace{-0.5cm}
\begin{multicols}{2}
    \begin{minipage}{\linewidth}
        \begin{itemize}
            \item [+] Gleichbleibender Platzbedarf, höhere Stromverstärkung
            \item [+] $ B \approx B_1 \cdot B_2 $ im Bereich <1000 
            \item [+] $ \beta \approx \beta_1 \cdot \beta_2 $ im Bereich <50'000
        \end{itemize}
    \end{minipage}
    
    \begin{minipage}{1.2\linewidth}
        \begin{itemize}
            \item [-] grosse Phasenverschiebung
            \item [-] für Hochfrequenzanwendungen ungeeingnet
            \item [-] langsame Schaltzeiten
            \item [-] doppelte Basis-Emitter-Spannung
        \end{itemize}
    \end{minipage}
\end{multicols}
Für effiziente Schaltanwendungen eignen sich Darlingtontransistoren wegen diesen Nachteilen kaum.

\subsection{MOSFET}
\begin{wrapfigure}{r}{7cm}
    \vspace{-1cm}
    \includegraphics[width=\linewidth]{images/mosfetprinz}
    \newline
    \includegraphics[width=\linewidth]{images/mosfetprak}
\end{wrapfigure}
Die elektrishe Leitfähigkeit des Substrats ist duch ein el. Feld gesteuert. Das el. Feld ruft im Substraht einee Influenzladung hervor.\newline
Die Gate-Elektrode ist durch ein Metaloxid vom Substraht isoliert.\newline\newline
S = Source \quad D = Drain \newline
G = Gate \quad B = Bulk(Substraht)\newline\newline
$ U_{DS} $ ist positiv damit ist der rechte pn-Übergang in Sperrrichtung gepolt. Deswegen kann keink Storm in beide Richtungen fliessen.\newline
$ \rightarrow $ Der Transistor ist selbstsperrend.\newline
\danger Sobald eine positive Spannung zwischen $ G $ und $ S $ angelegt ist, entsteht ein leitfähiger n-Kanal und damit auch ein Strom vom D- zum S-Anschluss.

\subsection{IGBT}
Der IGBT setzt such aus einem Bipolartransistor $ T_2 $ und einem MOSFET $ T_1 $ zusammen.\newline
    n- ist eine schwach dotierte Zone, welche zut Erhöhung der Spannungsfestigkeit verwendet wird.
\begin{center}
  \includegraphics[width=0.7\linewidth]{images/IGBTaufbau}
  \includegraphics[width=0.25\linewidth]{images/IGBTkennlinie}
\end{center}
\vspace{-0.5cm}
\subsubsection{Eigenschaften}
\begin{multicols}{2}
    \begin{itemize}
        \item Über die Kollektor-Emitter-strecke fällt mindestens die Schleusenspannug ab
        \item kleine Durchlassverluste bei hehen Ströme
        \item in Rückwärtsrichtung nur begrenzt Sperrfähig
        \item Grosse Sperrverluste vorallem beim Abchlaten
    \end{itemize}
\end{multicols}
\clearpage

\subsection{Transistoren im Vergleich}
\includegraphics[width=\linewidth]{images/transdiff}
%\begin{tabular}{lccc}
%    &\textbf{Darlington-Transistor}&\textbf{MOSFET}&\textbf{IGBT}\\
%    Schaltsymbol&
%    \includegraphics[width=1cm]{images/darlingtonSymbol}&
%    \includegraphics[width=2cm]{images/MOSFETSymbol}&
%    \includegraphics[width=1cm]{images/IGBTSymbol}\\
%    
%    Schichtaufbau&
%    \includegraphics[width=1cm]{images/darlingtonSchicht}&
%    \includegraphics[width=1cm]{images/MOSFETSchicht}&
%    \includegraphics[width=1cm]{images/IGBTSchicht}\\
%    
%    &&&\\
%    &&&\\
%    &&&\\
%    &&&\\
%    &&&\\
%    &&&\\    
%\end{tabular}

%=========================================
\clearpage
\begin{minipage}{0.7\linewidth}
\section{Thyristoren}
Ein Thyristor besteh aus vier Halbleiterschichten d.h. aus drei pn-Übergängen\newline
Thyrisotren sind einschaltbare Bauelemente.\newline
Thyristoren sind  \" einschaltbare Dioden\". Thyristoren werden mit dem Zündimpuls der Zwischen Gate (G) und Kathode (K) kurzzeitig anliegt durchgeschalten.
\end{minipage}
\begin{minipage}{0.3\linewidth}
     \includegraphics[width=0.5\linewidth]{images/thyraufbau}
\end{minipage}
\includegraphics[width=0.15\linewidth]{images/thyrESB}
\includegraphics[width=0.4\linewidth]{images/thyrKennlinie}
\includegraphics[width=0.45\linewidth]{images/thyrSchaltung}

\subsection{Thermische Eratzschaltung}
\begin{tabular}{ll}
	\textbf{Thermische Kenngrösse}			& \textbf{Elektrische Kenngrösse}\\
	Wärmeleistung P [W]						& Strom I  [A]\\
	Temperaturunterschied $ \vartheta $[K] 	& Spannung [V]\\
	Wärmewiderstand $ R_{th} $ {K/W}		& Widerstand (V/A)\\
\end{tabular}

\begin{multicols}{2}
	\begin{minipage}{\linewidth}
		\subsubsection{Thyrisor ohne Kühlung}
		\includegraphics[width=0.5\linewidth]{images/thyrOK}
		\[ \vartheta_{vJ}-\vartheta_U=P \cdot (R_{th\; JG}+R_{th\; GU}) \]		
	\end{minipage}

	\begin{minipage}{\linewidth}
		\subsubsection{Thyrisor mit Kühlung}
		\includegraphics[width=0.5\linewidth]{images/thyrMK}
		\[ \vartheta_{vJ}-\vartheta_U=P \cdot (R_{th\; JG}+R_{th\; GK}+R_{th\; KU})\qquad R_{th \; KU}=\dfrac{\Delta \vartheta}{P} \]	
	\end{minipage}
\end{multicols}

\begin{minipage}{0.5\linewidth}
    \subsection{Abschaltbarer Thyristor}
    \begin{minipage}{0.7\linewidth}        
        \textbf{(GTO = Gate-Turn-Off)}\newline
        Der GTO Schaltet aus, wenn ein ausreichend hoher nagativer Gate-Strom auftritt.\newline
        Amplitude des Gate-Stromes muss 20\% bis 30\% des abzuschaltenden GTO-Stromes betragen.
    \end{minipage}
    \begin{minipage}{0.2\linewidth}
        \includegraphics[width=\linewidth]{images/GTOSymbol}
    \end{minipage}    
\end{minipage}
\begin{minipage}{0.5\linewidth}
    \subsection{IGCT}
    \begin{minipage}{0.7\linewidth}
        \textbf{Integrated Gate-Commutated Thyristor}
        IGCT sind die Weiterentwicklung der GTO.\newline
        Sie werden hauptsächlich für Mittelspannungsumrichter iengesetzt.
    \end{minipage}
    \begin{minipage}{0.2\linewidth}
        \includegraphics[width=\linewidth]{images/IGCTSymbol}
    \end{minipage} 
\end{minipage}
\clearpage





















